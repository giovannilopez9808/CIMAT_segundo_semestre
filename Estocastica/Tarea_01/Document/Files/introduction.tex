\section{Introducción}

La comunicación humana por medio de redes de telefonía móvil y las conexiones inalambricas ha tenido un aumento en su uso día a día.\cite{Inzaurralde_2014} El uso masivo de este tipo de servicios creá un problema interdisiciplinario. Las ondas electromagnéticas que son usadas para la recepción y transmición de información pueden tener una interferencia debido a la cercania de las frecuencias que son usadas para crear estas redes de comunicación. El espectro electromagnético es un recurso límitado, por lo que la optimización del número y localización de las frecuencias es un problema a enfrentar para realizar un buen uso del mismo, incrementando la calidad de la comunicación.

El término del problema de asignación de frecuencias (FAP por sus siglas en inglés) ha sido usado para describir varios tipos de problemas, cada uno con sus objetivos y modelos diferentes. Estos problemas estan incluidos los siguientes:

\begin{enumerate}
    \item Planeación del frecuencias permanentes, las cuales maximizan el uso de todo el espectro electromagnético.\cite{Zoellner_1973}
    \item Planeación del diseño de modelos dada la localización de cada fuente.
    \item Asignación dinámica de frecuencias con una linea establecida de comunicación.
\end{enumerate}