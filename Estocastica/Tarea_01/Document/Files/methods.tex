
\section{Métodos y materiales}

La tarea que realizan los modelos de FAP es llevar al mínimo global la función de costo que depende el objetivo particular del problema. En este caso se utilizará una función de costo la cual contempla la interferencia provocada por la distancia cercana entre canales usados en cada nodo de la red. Sean dos nodos de comunicación $c_i$ y $c_j$ de una red $s$ donde $i\neq j$, entonces la función de costo esta dada por la ecuación \ref{eq:cost_function}.\cite{Lai_2015}

\begin{equation}
    Cost(s) = \sum_{|c_i-c_j| \leq d_{ij}} p_{ij} \label{eq:cost_function}
\end{equation}

donde $d_{ij}$ es la distancia máxima entre canales de los nodos donde sucede la interferencia y $p_{ij}$ es la penalización de la interferencia.

\subsection{Datos GSM2-272}

Los datos de las penalización y la distancia máxima entre canales fue obtenida a partir del artículo de Montemanni\cite{Montemanni_2010,data}. Los datos tienen el siguiente formato por columnas:

\begin{enumerate}
    \item \textbf{a}: índices del primer nodo.
    \item \textbf{b}: índices del segundo nodo.
    \item \textbf{R}: primer carácter de control.
    \item $>$ : segundo carácter de control.
    \item \textbf{s}: Distancia máxima entre canales en la que se contempla la interferencia.
    \item \textbf{p}: Penalización que se recibe si la distancia entre los canales es menor a la distancia máxima.
\end{enumerate}

El total de nodos es de 272 y el número total de lineas de conexión entre dos nodos es de 14525.

\subsection{Algoritmo}

La manera que se empleo la solución del problema fue siguiendo el algoritmo \ref{alg:FAP}. Al inicio se contemplo la lectura total de los datos de la red GSM2-272 en una clase llamada lines, la cual contiene una lista donde cada elemento contiene la información de cada conexión entre dos nodos. En seguida se realizan 100,000 iteraciones en donde cada iteración se le asigna a cada nodo un canal aleatorio entre 1 y $n_c$, donde $n_c$ es el número de canales a asignar. Con la asignación realizada se calcula la función de costo para la red. Se guardo un resultado por cada 100,000 iteraciones realizadas.

\begin{algorithm}[H]
    \caption{}
    \label{alg:FAP}
    \KwInput{n\_channels ,seed}
    \KwOutput{minimum}
    n\_nodes $\gets$ 272\\
    n\_lines $\gets$ 14525\\
    lines $\gets$ GSM2\_data\\
    minimum $\gets$ MAX\_int\\
    \For{$n=1,100000$}
    {
        Cost $\gets$  0\\
        \For{k=1,n\_nodes}
        {
            node$_k$ $\gets$ random\_channel(n\_channels)\\
        }
        \For{k=1,n\_lines}
        {
            c$_i$, c$_j$, d$_{ij}$, p$_{ij}$ $\gets$ get\_information(lines[k])\\
            \If{$|$c$_i$-c$_j|\leq$d$_{ij}$}{
                cost $\gets$ cost + p$_{ij}$\\
            }
        }
        \If{minimum$>$cost}{
            minimum $\gets$ cost
        }
    }
    save\_result(minimum)\\
\end{algorithm}

El algoritmo \ref{alg:FAP} ejecutaron 100 veces en paralelo con 34, 39 y 49 canales consecutivos disponibles en cada nodo. Cada ejecucción tuvo una semilla diferente que va desde la 1 hasta la 100.