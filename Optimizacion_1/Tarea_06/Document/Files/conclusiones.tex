\section{Conclusiones}


El método de descenso del gradiente con el tamaño de paso obtenido a partir del algoritmo de bisección es bueno para este conjunto de datos seleccionado debido a que llegan a un óptimo global de la función con un pequeño número de iteraciones comparado con el tamaño de valores diferentes que se dio como entrenamiento. A pesar que en algunos puntos el gradiente parecia tomar valores oscilantes estos ayudaron para que el punto fuera encontrado de forma exitosa. Se intento utilizar el algoritmo de back traking pero este tardo alrededor de 2 días en llegar al mismo punto que el algoritmo de bisección.