\section{Resultados}

Se ejecutaron 30 veces cada función con diferente método. Cada ejecucción partio desde el punto

\begin{equation*}
    x_0 = x^* + \mathcal{U}(-2,2)
\end{equation*}

Donde el punto óptimo para era dado para cada función. El promedio del tiempo de ejecucción, valor de la función en la última iteración, el valor del gradiente en la última y el número de iteraciones se encuentran en las tablas \ref{table:time}, \ref{table:function}, \ref{table:gradient} y \ref{table:iterations} respectivamente.

\begin{table}[H]
    \centering
    \begin{tabular}{lrrrr}
        \hline
        \textbf{Funcion} & \textbf{Newton-Cauchy} & \textbf{Newton modificado} & \textbf{Dogleg} & \textbf{Cauchy} \\
        \hline
        Wood             & 1.28845                & 0.027438                   & 0.330035        & 1.85432         \\
        Rosembrock       & 74.3634                & 1.58818                    & 41.6005         & 44.9421         \\
        Branin           & 0.0171616              & 0.00662657                 & 0.0159199       & 0.0343389       \\
        \hline
    \end{tabular}
    \caption{Tiempo promedio en segundos de las ejecucciones realizadas con cada función y método implementado.}
    \label{table:time}
\end{table}

\begin{table}[H]
    \centering
    \begin{tabular}{lrrrr}
        \hline
        \textbf{Funcion} & \textbf{Newton-Cauchy} & \textbf{Newton modificado} & \textbf{Dogleg} & \textbf{Cauchy} \\
        \hline
        Wood             & 7.00204e-08            & 0.201103                   & 2.65275         & 0.00010695      \\
        Rosembrock       & 1.19599                & 19090.9                    & 32.0271         & 0.39945         \\
        Branin           & 0.397888               & 0.648666                   & 0.397887        & 0.397889        \\
        \hline
    \end{tabular}
    \caption{Promeio del valor de la función en la última iteración de las ejecucciones realizadas con cada función y método implementado.}
    \label{table:function}
\end{table}

\begin{table}[H]
    \centering
    \begin{tabular}{lrrrr}
        \hline
        \textbf{Funcion} & \textbf{Newton-Cauchy} & \textbf{Newton modificado} & \textbf{Dogleg} & \textbf{Cauchy} \\ \hline
        Wood             & 0.00190684             & 1.08682                    & 0.343427        & 0.0198407       \\
        Rosembrock       & 0.00176435             & 6112.75                    & 1.9312          & 0.0430613       \\
        Branin           & 0.000997418            & 0.278091                   & 0.000877879     & 0.00262148      \\
        \hline
    \end{tabular}
    \caption{Promedio del valor del gradiente en la última iteración de las ejecucciones realizadas con cada función y método implementado.}
    \label{table:gradient}
\end{table}

\begin{table}[H]
    \centering
    \begin{tabular}{lrrrr}
        \hline
        \textbf{Funcion} & \textbf{Newton-Cauchy} & \textbf{Newton modificado} & \textbf{Dogleg} & \textbf{Cauchy} \\
        \hline
        Wood             & 1044                   & 15.4667                    & 258.467         & 1538.7          \\
        Rosembrock       & 6623.5                 & 63.3                       & 4912.67         & 6845.4          \\
        Branin           & 11.1                   & 3.66667                    & 10.2            & 22.5            \\
        \hline
    \end{tabular}
    \caption{Iteraciones promedio de las ejecucciones realizadas con cada función y método implementado.}
    \label{table:iterations}
\end{table}