\section{Introducción}

La optimización es procedimiento que con sus resultados se toman de decisiones en el análsis de sistemas físicos. Para realizar es la optimización de un sistema o situación se debe de contemplar un objetivo, el cual debe ser caracterizado por una función cuantitativa.

El resultado de realizar la optimización a un sistema puede representarse como un ahorro de tiempo, energía o cualquier objeto que pueda ser reflejado en un número. El objetivo del proceso de una optimización es obtener un conjunto de números o caracteristicas que representen un mínimo o máximo de objeto el cual esta siendo caracterizado. Esto puede ser representado como se encuentra en la ecuación \ref{eq:definition_optimization}.

\begin{equation}
    \underset{x\in \mathbb{R}}{min} \; f(x)
    \label{eq:definition_optimization}
\end{equation}

donde $f$ es la caracterización cuantitativa del problema y $x$ son los objetos que interactuan con el sistema.

Una optimización local es una solución al problema de optimización en una vecindad alrededor del valor de $x$ encontrado. En cambio una optimización global es aquella solución que es menor o mayor con respecto a todas las demás. La solución de un proceso de optimización no siempre encontrará los valores en que el sistema se situe una optimización global.