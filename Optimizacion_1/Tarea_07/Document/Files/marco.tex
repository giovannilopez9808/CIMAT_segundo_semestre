\section{Marco teórico}

Una técnica para la optimización de funciones es definir la función de objetivo, generalmente para obtener alguna solución del problema se escogen pasos de búsqueda unidireccionales.

\subsection{Región de confianza}

Un método empleado para acotar las soluciones del problema es usar una región de confianza dado un punto. Para formalizar este método se define un modelo que aproxima a la función objetivo en un punto. El modelo se encuentra descrito en la ecuación \ref{eq:quadratic_model}.

\begin{equation}
    m_k (p) = f(x_k) + \nabla f(x_k)^T p + \frac{1}{2} p^T \nabla^2 f(x_k) p
    \label{eq:quadratic_model}
\end{equation}

donde su mínimo se encuentra dentro de la región de confianza. Se denomina al arco de Cauchy al segmento tal que cumple la ecuación \ref{eq:cauchy_arc}.

\begin{equation}
    x_k^C(t) = \{x: x=x_k + t\nabla f(x_k), t\leq 0, ||t\nabla f(x_k)|| < \Delta_k \}
    \label{eq:cauchy_arc}
\end{equation}

El punto de Cauchy se encuentra definido en la ecuación \ref{eq:cauchy_point}.

\begin{equation}
    x_k^C = x_k + t_k^C \nabla f(x_k) = \underset{\tiny \begin{matrix}t\geq 0 \\ x_k + t \nabla f(x_k) \in B(x_k,\Delta_k) \end{matrix}}{arg\; min} m_k(t\nabla f(x_k)) \label{eq:cauchy_point}
\end{equation}

