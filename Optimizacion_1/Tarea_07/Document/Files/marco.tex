\section{Marco teórico}

Una técnica para la optimización de funciones es definir la función de objetivo, generalmente para obtener alguna solución del problema se escogen pasos de búsqueda unidireccionales.

\subsection{Región de confianza \label{sec:trust_region}}

Un método empleado para acotar las soluciones del problema es usar una región de confianza dado un punto. Para formalizar este método se define un modelo que aproxima a la función objetivo en un punto. El modelo se encuentra descrito en la ecuación \ref{eq:quadratic_model}.

\begin{equation}
    m_k (p) = f(x_k) + \nabla f(x_k)^T p + \frac{1}{2} p^T \nabla^2 f(x_k) p
    \label{eq:quadratic_model}
\end{equation}

donde su mínimo se encuentra dentro de la región de confianza.

Une medida para calcular el radio de la región de confianza  $\Delta_k$ es la medida de ajuste que esta definida en la ecuación \ref{eq:adjustment_equation}.

\begin{equation*}
    \rho_k = \frac{f(x_k)-f(x+p_k)}{m_k(0)+m_k(p_k)}  \label{eq:adjustment_equation}
\end{equation*}

Donde el numerador representa la reducción en la función y el denominador la reducción en el modelo. El denominador siempre es positivo debido a que $p_k$ minimiza el modelo en cada iteración. En el caso en que $\rho_k$ sea un valor negativo este representa un incremento en la función $f$. Por lo que el paso deberá rechazarse. Si $\rho_k$ es muy cercano a uno, entonces el comportamiento de la función $f$ y el modelo $m_k$ tienen una semejanza, por lo que se optaría a incrementar la región de confianza. Si $\rho_k$ se encuentra entre 0 y 1, entoces se opta por no realizar modificaciones en la región de confianza. Si $\rho_k$ es cercano a cero o negativo se propone una reducción en la región de confianza.

\subsection{Punto de Cauchy}

Se denomina al arco de Cauchy al segmento tal que cumple la ecuación \ref{eq:cauchy_arc}.

\begin{equation}
    x_k^C(t) = \{x: x=x_k + t\nabla f(x_k), t\leq 0, ||t\nabla f(x_k)|| < \Delta_k \}
    \label{eq:cauchy_arc}
\end{equation}

El punto de Cauchy se encuentra definido en la ecuación \ref{eq:cauchy_point}.

\begin{equation}
    p_k^S = \argmin{||p||<\Delta_k} m_k(p) \label{eq:cauchy_point}
\end{equation}

donde $\Delta_k$ es la región de confianza.

Teniendo el punto $p_k^S$, se buscará un parámetro $\tau_k$, el cual mínimize el modelo de la ecuación \ref*{eq:quadratic_model} en la región de confianza (ecuación \ref{eq:tau_cauchy}).

\begin{equation}
    \tau_k = \argmin{\tau_k \geq 0} m_k(\tau p_k^S) \leq \Delta_k \label{eq:tau_cauchy}
\end{equation}

por lo que el punto de Cauchy, lo estaremos calculando de tal manera que $p_k^C = \tau_k p_k^S$.

Tomando a $p_k$ como $-\lambda_k g_k$, sujeto a la región de confianza, se tiene que

\begin{equation*}
    ||p|| = ||-\lambda_k g_k|| \Rightarrow \lambda_k \leq \frac{\Delta_k}{||g_k||}
\end{equation*}


Por lo que la ecuación \ref{eq:tau_cauchy} se puede escribir como en la ecuación .

\begin{equation}
    \lambda_k = \argmin{\lambda \in [0,\hat{\lambda}]}  m(-\lambda g_l) \label{eq:lambda_cauchy}
\end{equation}

La solución al problema planteado en la ecuación \ref{eq:lambda_cauchy} se encuentra descrita en la ecuación \ref{eq:lambda_k_cauchy}.

\begin{equation}
    \lambda_k = \hat{\lambda} \left\{\begin{matrix}
        1                                                            & \text{si } g^T_k B_k g_k      \\
        min\left(1, \frac{||g_k||^3}{\Delta_k g_k^T B_k g_k} \right) & \text{en cualquier otro caso}
    \end{matrix}\right. \label{eq:lambda_k_cauchy}
\end{equation}

Por lo tanto, el punto de Cauchy se obtiene como en la ecuación \ref{eq:cauchy_p_k}.

\begin{equation}
    p_k^C = -\tau_k \frac{\Delta_k}{||g_k||} g_k \label{eq:cauchy_p_k}
\end{equation}

\section{Método Dogleg}