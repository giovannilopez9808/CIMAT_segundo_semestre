\section{Conclusiones}

Con los resultados anteriormente expuestos se pueden llegar a las siguientes conclusiones:

\begin{itemize}
    \item El método de Newton modificado tiene tiempos de ejecucción y número de iteraciones menores a comparación del resto de métodos independientemente de la función que se uso.
    \item El método de Cauchy y Newton-Cauchy logran obtener el mínimo global de la función con mayor frecuencia.
    \item El método de Newton modificado tiende a llegar a mínimos locales pero en ocasiones si llega a obtener el mínimo global de la función.
    \item Para funciones que tienen una dimensión alta en su hessiano, el método de Newton-Cauchy y Cauchy llegan a obtener un número grande de iteraciones y tiempo promedio.
    \item Dogleg al ser una combinación de los métodos llega a presentar caracteristicas antes mencionadas.
    \item El tamaño de $\Delta_k$ hace varian mucho el resultado, en algunos casos los métodos entraban en un ciclo sin fin si el $\Delta_k$ era muy grande $\Delta_k>1$ o muy pequeño $\Delta_k < 0.001$, pero siempre sobre un mínimo local. Esto pasaba con frecuencia en la función de Rosembrock y Branin.
\end{itemize}