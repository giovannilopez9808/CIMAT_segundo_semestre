\section*{Problema 04}

\textbf{Sea}

\begin{equation*}
    f(x) = \frac{1}{2}x^T \begin{bmatrix}
        \frac{3}{2} & 2 \\
        0 \frac{3}{2}
    \end{bmatrix}x + x^T \begin{bmatrix}
        3 \\ -1
    \end{bmatrix} -22
\end{equation*}

\subsection*{Parte 1}

\textbf{Si se usa al algoritmo de gradiente descendente con tamaño de paso fijo para mininizar la funcion anterior, diga el rango de valores que puede tomar el tamaño de paso para que el algoritmo converja al minimizador.}

\subsection*{Parte 2}

\textbf{Calcula el tamaño de paso exacto $\alpha_0$ si el punto incial es $x_0 = [0, 0]^T$?}

Se tiene que se puede calcular el k-esimo tamaño de paso de la siguiente manera:

\begin{equation*}
    \alpha_k = \frac{g_k^Tg_k}{g_k^TQg_k}
\end{equation*}

En nuestro caso, el gradiente es:

\begin{equation*}
    g_k = Qx-b
\end{equation*}

para $x_0 = [0, 0]^T$, se tiene que $g_k$ es

\begin{equation*}
    g_k = \begin{bmatrix}
        -3 \\ 1
    \end{bmatrix}
\end{equation*}

por ende

\begin{equation*}
    g_k^T g_k = 10 \qquad g_k^TQg_k = 9
\end{equation*}

por lo tanto el tamaño de paso para $x_0$ es :

\begin{equation*}
    \alpha_k = \frac{10}{9}
\end{equation*}