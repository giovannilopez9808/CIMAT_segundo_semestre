\section*{Problema 02}

\textbf{Sea A una matriz positiva definida. Muestra que}

\begin{equation*}
    A_{ij} < \frac{A_{ii}+A_{jj}}{2}
\end{equation*}


Sea $A'$ una matriz tal que sus elementos son submatrices de la matriz A tal que:

\begin{equation*}
    A' = \begin{bmatrix}
        A_{ii} & A_{ij} \\
        A_{ji} & A_{jj}
    \end{bmatrix}
\end{equation*}

calculando el determinante de $A'$ se tiene lo siguiente:

\begin{equation*}
    det A' = A_{ii}A_{jj} - A_{ij}^2
\end{equation*}

lo cual es una descomposición de Schur de la matriz A. Como A es una matriz positiva definida entonces

\begin{equation*}
    A_{ii}A_{jj} - A_{ij}^2 > 0
\end{equation*}

por ende

\begin{equation*}
    A_{ij} < \sqrt{A_{ii}A_{jj}}
\end{equation*}

por la desigualdad de $\sqrt{A_{ii}A_{jj}} \leq \frac{A_{ii}+A_{jj}}{2}$ entonces:

\begin{equation*}
    A_{ij} < \frac{A_{ii}+A_{jj}}{2}
\end{equation*}