\section*{Problema 03}

\textbf{Show that if functions $f : R^n \rightarrow R$ and $g : R^n \rightarrow R$ satisfy $f (x) = -g(x) + o(g(x))$ and $g(x) > 0$ for all $x \neq 0$, then for all $x \neq 0$ sufficiently small, we have $f (x) < 0$.}

Siguiendo la definición de la notació o pequeña, se tiene que:

\begin{equation*}
    o(g(x)) = \limit{x}{0} \frac{h(x)}{g(x)}
\end{equation*}

entonces podemos definir que $h(x)=g(x)e(x)$, donde

\begin{equation*}
    \limit{x}{0} e(x) =0
\end{equation*}

Con esto, la función $f(x)$ puede ser escrita de la siguiente manera:

\begin{align*}
    f(x) & = -g(x) + e(x)g(x) \\
    f(x) & =  g(x) (e(x)-1)
\end{align*}

Calculando el límite cuando $f(x)$ tiende a cero, se obtiene lo siguiente:

\begin{align*}
    \limit{x}{0} f(x) & = \limit{x}{0} g(x) (e(x)-1)                                          \\
                      & =\limit{x}{0} g(x) \limit{x}{0} (e(x)-1)                              \\
                      & =\limit{x}{0} g(x) \left (\limit{x}{0} e(x) - \limit{x}{0} -1\right ) \\
                      & =\limit{x}{0} g(x) \left (-1\right )
\end{align*}

como $g(x)>0 \forall x\neq 0$, entonces, se puede obtener el límite cuando x tiende a 0 es positivo, por lo tanto:

\begin{align*}
    \limit{x}{0} f(x) & = \lim{x}{0} -g(x) \\
    \limit{x}{0} f(x) & <0
\end{align*}