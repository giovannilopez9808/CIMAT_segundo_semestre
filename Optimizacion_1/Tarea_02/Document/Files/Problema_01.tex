\section*{Problema 01}

\textbf{Show that $x-sinx = o(x^2)$, as $x\rightarrow 0$}

Realizando el límite cuando $x\rightarrow 0$ de $f(x)=\frac{x-sinx}{x^2}$, se obtiene lo siguente:

\begin{align*}
	\underset{x\rightarrow 0}{lim} f(x) & = \underset{x\rightarrow 0}{lim} \frac{x-sinx}{x^2}
\end{align*}

aplicando L'Hopital a este límite se encuentra que:

\begin{align*}
	\underset{x\rightarrow 0}{lim} \;f(x) & = \underset{x\rightarrow 0}{lim} \;\;\frac{x-sinx}{x^2} \\
	                                      & = \underset{x\rightarrow 0}{lim} \;\;\frac{1-cosx}{x}   \\
	                                      & = \underset{x\rightarrow 0}{lim} \;\;sinx               \\
	\underset{x\rightarrow 0}{lim} \;f(x) & =
\end{align*}


como obtenemos que el límite cuando $x\rightarrow 0$ de $f(x)$ tiende a 0, entonces

\begin{equation*}
	x-sinx \in o(x^2)
\end{equation*}
