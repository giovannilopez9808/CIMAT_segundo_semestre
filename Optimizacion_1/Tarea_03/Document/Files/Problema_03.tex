\section*{Problema 03}

\textbf{Show that the following function $f: \mathbb{R}^n \rightarrow \mathbb{R}$ is convex.}

\begin{equation*}
    f(x) = -exp(-g(x))
\end{equation*}

\textbf{where $g:\mathbb{R}^n \rightarrow \mathbb{R}$ has convex domain and satisfaces}

\begin{equation*}
    \begin{bmatrix}
        \nabla^2 g(x) & \nabla g(x) \\
        \nabla^T g(x) & 1
    \end{bmatrix} \succeq 0
\end{equation*}

\textbf{for $x \in \text{dom } g$}

Calculando el gradiente de $f$, se obtiene lo siguiente:

\begin{align*}
    \nabla f & = \dpartial{f}{g(x)} \nabla g \\
    \nabla f & = e^{-g(x)} \nabla g
\end{align*}

Calculando el hessiano de $f$, se obtiene lo siguiente:

\begin{align*}
    \nabla^2 f & = -e^{-g(x)} \nabla g \nabla g^T + e^{-g(x)} \nabla^2 g   \\
               & = e^{-g(x)} \left (\nabla^2g -\nabla g\nabla g^T \right ) \\
\end{align*}

donde por el complemento de Schur se tiene que $\nabla^2g -\nabla g\nabla g^T \succeq 0$, y  como $e^{-g(x)}>0$. Entonces $\nabla^2 f>0$, por lo tanto $f$ es convexa.