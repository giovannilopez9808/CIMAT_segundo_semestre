\section*{Problema 09}

\textbf{Let $f : \mathbb{R}^n \rightarrow \mathbb{R}$ be a differentiable function. Show that if f is convex over a nonempty convex set C then}

\begin{equation*}
    (\nabla f(x)-\nabla f(y))^T (x-y) \geq 0, \qquad \forall x,y \in C
\end{equation*}


Supongamos que se cumple:

\begin{equation*}
    (\nabla f(x)-\nabla f(y))^T (x-y) < 0, \qquad \forall x,y \in C
\end{equation*}

Se tiene que $x+\alpha(y-x) \in C$  si $x,y\in C$ con $\alpha \in [0,1]$, entonces:

\begin{align*}
    (\nabla f(x+\alpha(y-x)) - \nabla f(x))^T (x+\alpha (y-x)) & < 0 \\
    \alpha (\nabla f(x+\alpha(y-x)) - \nabla f(x))^T (y-x)     & < 0
\end{align*}

por consiguiente

\begin{equation*}
    \nabla f(x+\alpha (y-x))  < \nabla f(x)
\end{equation*}

nombrando a

\begin{equation*}
    \theta (\alpha) = f(x+\alpha (y-x))
\end{equation*}

la cual

\begin{equation*}
    \theta(0) = f(x) \qquad \theta(1) = f(y)
\end{equation*}

calculando el gradiente de la función $\theta$, se obtiene que

\begin{equation*}
    \nabla \theta (\alpha) =  \nabla f (x+\alpha (y-x))^T (y-x)
\end{equation*}

usando el teorema fundamental del calculo se tiene lo siguiente:

\begin{align*}
    f(y)-f(x) & = \theta(1) - \theta(0)                            \\
              & = \int_0^1 \nabla\theta(\alpha) d\alpha            \\
              & = \int_0^1 \nabla f(x+\alpha (y-x))^T(y-x) d\alpha \\
              & < \int_0^1 \nabla f(x)^T (y-x) d\alpha             \\
              & < \nabla f(x)^T(y-x) \int_0^1 d\alpha              \\
              & < \nabla f(x)^T(y-x)
\end{align*}

por lo tanto:

\begin{equation*}
    f(y)-f(x) < \nabla f(x)^T(y-x)
\end{equation*}

lo cual contradice que la función $f$ es convexa. Por ende la suposición que se realizo es incorrecta. Por ende, si $f$ es convexa, entonces se cumple que

\begin{equation*}
    (\nabla f(x)-\nabla f(y))^T (x-y) \geq 0, \qquad \forall x,y \in C
\end{equation*}