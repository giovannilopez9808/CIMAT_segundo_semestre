\section{Marco teórico}

\subsection{Convexidad}

El concepto de convexidad en los problemas de optimización es de gran importancia. Este concepto aporta una mayor facilidad al resolver un problema. La convexidad puede ser aplicado a conjunto o funciones. Se dice que $S$ es un conjunto convexo si un segmento de linea conecta cualquier par de puntos en $S$. Sean $x,y \in S$, entonces $\alpha x + (1-\alpha)y \in S$ donde $\alpha \in [0,1]$. Se dice que una función es convexa si su dominio $S$ es convexo y cumplen con la propiedad escrita en la ecuación .

\begin{equation}
    f(\alpha x + (1-\alpha)y) \leq \alpha f(x) + (1-\alpha)f(y)
\end{equation}