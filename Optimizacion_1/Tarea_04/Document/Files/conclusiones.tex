\section{Conclusiones}

Con los resultados expuestos anteriormente se pueden obtener las siguientes conclusiones:

\begin{itemize}
    \item El método de Newton realiza una cantidad menor de iteraciones a comparación dle método del descenso del gradiente. Esto es debido a la elección continua del tamaño de paso que se calcula en cada iteración. Esto es reflejado en las tablas \ref{table:rosembrock_2}, \ref{table:rosembrock_2_random}, \ref{table:rosembrock_100}, \ref{table:rosembrock_100_random}, \ref{table:wood_predefined} y \ref{table:wood_random}
    \item El método del descenso del gradiente logra obtener con mayor frecuencia un mínimo global. Si se obtuvo un mínimo local este no tiene un valor muy lejano al mínimo global. Esto puede verse en las tablas \ref{table:rosembrock_2_random_newton}, \ref{table:rosembrock_100_random_gradient}, \ref{table:rosembrock_100_random_newton}, \ref{table:rosembrock_100_random_gradient}, \ref{table:wood_4_random_newton} y \ref{table:wood_4_random_gradient}. En cambio el método de Newton tiende a detenerse en mínimos locales que esten cercanos a el punto inicial dado.
\end{itemize}