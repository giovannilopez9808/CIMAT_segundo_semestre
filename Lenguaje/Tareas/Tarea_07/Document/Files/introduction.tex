\section{Introduction}

The sentiment analysis or mining opinion is the computational study about opinions, comments, emotions, feelings and actitude
El análisis de sentimiento o opinion mining es el estudio computacional de las opiniones, comentarios, emociones, sentimientos y attitudes about entities as services, organizations, people, problems, events and topics\cite{Liu_2015}. The growth of the study of the sentiments in documents matchs with the social networks boom. The blogs, forums and twitter are include in this socialm networks. The sentiment analysis is one of the most active fields about Natural Language Processing (NLP). This tecnic has been used in other knowledge areas as managment sciende and social science as marketing, finance, politic science, communication and history\cite{Chalothom_2015}.

There are many methods that perform the sentiment analysis. This methods involves supervised learning and non supervised learning. The non supervised learning methods take advantage of lexicons, gramatical and syntactic patterns. The support vector machines (SVM), maximum entropy and naive bayes are examples for non supervised methods\cite{Liu_2012,Liu_2015,Lee_2009}.

At the beginning of the 2010s, Deep learning surges as an important method for the supervised learning\cite{Goodfellow_2016}. This method produce results commensurable with the state-of-art on several applications in computer vision and speech recognition.

\subsection{Sentiment analysis tasks}

The sentiment analysis are divided in some levels\cite{Thomas_2013}.

\subsubsection{Document-level sentiment classification}

In this level, the document is trated as primary information. The document is classificated with negative or positive label. Yang\cite{Yang_2016} proposed a hierarchical attention network model that focus in the information to build a document representation. The biggest challenge in document-level sentiment classification is create relationships with the words on a extense corpus. This problem can be trated with a SR-LSTM\cite{Guozheng_2018}. SR-LSTM model is composed with a LSTM that learn each word vector (sentence vectors) and a second layer enconde the link between words.

\subsubsection{Sentence-level sentiment classification}

The setence-level is used to classify if one word is objective or subjective. One word is classificated with subjective label if this word don't contain any sentiment. Zhao\cite{Zhao_2018_weakly} proposed a framework called Weakly-supervised Deep Embedding (WDE). This framework was trained with review ratings and it's purpose to sentiment classifier using a Convolutional Neural Network (CNN). The architecture of this framework is two networks, WDE-CNN y (WDE-LSTM) to extract the review's vector representation. This model was evaluate with Amazon dataset from three domains (digital cameras, cell phones and laptops). The accuracy obtained on WDE-CNN model was 87.7\%, and on WDE-LSTM model was 87.9\%, which shows that deep learning models gives highest accuracy as compared to baseline models.

\subsubsection{Aspect-level sentiment classification}


Aspect level sentiment analysis is commonly called feature-based sentiment analysis or
entity-based sentiment analysis. This sentiment analysis task includes the identification of
features or aspects in a sentence (which is a user-generated review of an entity) and categorizing the features as positive or negative. The sentiment-target pairs are first identified, then they are classified into different polarities, and finally, sentiment values for every aspect are clubbed. Recently, attention-based LSTM mechanisms are being used for aspect-based sentiment analysis. Ma et al.\cite{Ma_2018} proposed a two-step attention architecture, which attends words of the target expression along with the whole sentence. The author also applied extended LSTM, which can utilize external knowledge for developing a common-sense system for target aspect-based sentiment analysis. The initial systems were not able to model different aspects in a sentence and do not explore the explicit position context of words. Hence, Ma et al.\cite{Ma_2019} developed a two-stage approach that can handle the above problems. In Stage-1, position attention model is introduced for modelling the aspects and its neighboring context words. In Stage-2 multiple aspect terms within a sentence are modelled simultaneously. The most recent approach is proposed by Yang et al.\cite{Yang_2019a}, which replaces the conventional attention models with coattention mechanism by introducing a Coattention-LSTM network that can model the context-level and target-level attention alternatively by learning the non-linear representations of the target and context simultaneously. Thus, the proposed model can extract more effective sentiment features for aspect-based sentiment analysis.

\subsubsection{Multi-domain sentiment classification}

The word domain is referred as a set of documents that are related to a specific topic. Multi-domain sentiment classification focuses on transferring information from one domain to the next domain. The models are first trained in source domain. The knowledge is then transferred and explored in another domain. Yuan et al.\cite{Yuan_2018} proposed a Domain Attention Model (DAM) for modeling the feature-level tasks using attention mechanism for multi-domain sentiment classification. DAM is composed of two modules: domain module and sentiment module. The domain mod- ule predicts the domain in which text belongs using bi-LSTM, and sentiment module selects the important features related to the domain using another bi-LSTM with attention mechanism. The vector thus obtained from the sentiment module is fed into a softmax classifier to predict the polarity of the texts. The author used Amazon multi-domain dataset containing reviews from four domains, and Sanders Twitter Sentiment dataset containing tweets about four different IT companies. The proposed model was compared with traditional machine learning approaches, and results show that the model performed well for multi-domain sentiment classification.

\subsubsection{Multimodal sentiment classification}

Different people express their sentiments or opinions in different ways. Earlier, the text was considered as the primary medium to express an opinion. This is known as a unimodal approach. With the advancement of technology and science, people are now shifting towards visual and audio modalities to express their sentiments. Combining or fusing more than one modalities for detecting the opinion is known as multimodal sentiment analysis. Hence, researchers are now focusing on this direction for improving the sentiment classification process. Poria et al.\cite{Poria_2016} proposed a novel methodology for merging the affective information extracted from audio, visual, and textual modalities. They discussed how different modalities were combined together to improve the overall sentiment analysis process. The experimental results showed that bimodal and trimodal models
have shown better accuracy as compared to unimodal models, which shows the importance
of using features from all the modality for enhancing the performance of sentiment analysis
models.

\subsubsection{Taxonomy of sentiment analysis}

Research in the field of sentiment analysis is taking place for several years. Initially, handcrafted features were used for various classification tasks. On the other hand, machine-learned features can be categorized into traditional machine learning-based approaches and deep learning-based approaches. Machine learning-based methods include Support Vector Machine (SVM), Naïve Bayes (NB), Maximum Entropy (ME), Decision tree learning, and Random Forests. They are further categorized into supervised and unsupervised learning methods.