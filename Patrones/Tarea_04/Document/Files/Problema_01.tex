\section*{Problema 01}

\textbf{Supongamos que (X,Y) son variables aleatorias discretas con la siguiente distribución conjunta:}

\begin{table}[H]
    \centering
    \begin{tabular}{ccccc} \hline
            & X=1  & X=2  & X=3  & X=4  \\  \hline
        Y=0 & 0.1  & 0.05 & 0.05 & 0.15 \\
        Y=1 & 0.12 & 0.1  & 0.25 & 0.18 \\ \hline
    \end{tabular}
\end{table}

\textbf{Queremos predecir Y en base del valor observado para X.}

\begin{itemize}
    \item \textbf{Calcula el clasificador Bayesiano Optimo si equivocarse de categoría tiene costo 1 y no equivocarse tiene costo 0. ¿Cuál es el costo (error) promedio para este clasificador?}
    \item \textbf{Calcula el clasificador Bayesiano Optimo si clasificar una observación mal cuando el verdadero valor es Y = 1 tiene un costo 3 y en el otro caso tiene costo 2.}
\end{itemize}

Para una x fija buscamos la asignación $\hat{Y}(x)$ que minimiza el error mostrado en la ecuación \ref{eq:error}.

\begin{equation}
    E_{Y|X=x} [L(Y,\hat{Y}(x))] \label{eq:error}
\end{equation}

Para este problema se tiene que

\begin{equation*}
    E_{Y|X=x} [L(Y,\hat{Y}(x))] = L(0,\hat{Y}(x)) P(Y=0|X=x) + L(1,\hat{Y}(x)) P(Y=1|X=x)
\end{equation*}

con $L(Y,Y)=0$.

\begin{align*}
    \text{Si } \hat{Y}(x)=0 & \Rightarrow E_{Y|X=x} [L(Y,\hat{Y}(x))]  = L(1,0)P(Y=1|X=x) \\[0.25cm]
    \text{Si } \hat{Y}(x)=1 & \Rightarrow E_{Y|X=x} [L(Y,\hat{Y}(x))]  = L(0,1)P(Y=1|X=x)
\end{align*}

Si

\begin{equation*}
    \frac{L(1,0)P(Y=1|X=x)}{L(0,1)P(Y=1|X=x)} > 1
\end{equation*}

entonces $\hat{Y}(x)=1$ minimiza el error. De lo contrario $\hat{Y}(x)=0$ obtiene el mínimo. Si el cociente es igual a 1 entonces las dos opciones minimizan. En particular elegimos $\hat{Y}(x)=0$.

Por lo tanto el clasificador tiene la siguiente forma:

\begin{align*}
    \hat{Y}(x) & = \mathbb{I} \left [\frac{L(1,0)P(Y=1|X=x)}{L(0,1)P(Y=1|X=x)} > 1 \right ]
    \hat{Y}(x) & = \mathbb{I} \left [\frac{P(Y=1|X=x)}{P(Y=1|X=x)} > \frac{L(0,1)}{L(1,0)} \right ]
\end{align*}

por el teorema de bayes, se obtiene lo siguiente:

\begin{align*}
    \hat{Y}(x) & = \mathbb{I} \left [\frac{P(Y=1|X=x)}{P(Y=1|X=x)} > \frac{L(0,1)}{L(1,0)} \right ]                                           \\
               & = \mathbb{I} \left [\frac{\frac{P(X=x|Y=1)P(Y=1)}{P(X=x)}}{\frac{P(X=x|Y=0)P(Y=0)}{P(X=x)}} > \frac{L(0,1)}{L(1,0)} \right ] \\
               & = \mathbb{I} \left [ \frac{P(X=x|Y=1)}{P(X=x|Y=0)}  > \frac{L(0,1)P(Y=0)}{L(1,0)P(Y=1)}\right ]
\end{align*}

Usando la ley de la probabilidad total, se tiene el siguiente resultado

\begin{align*}
    P(Y=0) & = \sum P(Y=0|X=x)P(X=x) = \sum P(Y=0,X=x) = 0.35 \\
    P(Y=1) & = \sum P(Y=1|X=x)P(X=x) = \sum P(Y=1,X=x) = 0.65
\end{align*}

Calculando las probabilidades condicionales se tiene lo siguiente

\begin{align*}
    \pxy{1}{1}=\frac{12}{65} & \qquad \pxy{1}{0}=\frac{10}{35} \\
    \pxy{2}{1}=\frac{10}{65} & \qquad \pxy{2}{0}=\frac{5}{35}  \\
    \pxy{3}{1}=\frac{25}{65} & \qquad \pxy{3}{0}=\frac{5}{35}  \\
    \pxy{4}{1}=\frac{18}{65} & \qquad \pxy{4}{0}=\frac{15}{35}
\end{align*}

por lo tanto, los coeficientes son

\begin{align*}
    \frac{\pxy{1}{1}}{\pxy{1}{0}} = \frac{420}{650} & \qquad \frac{\pxy{1}{1}}{\pxy{1}{0}} = \frac{875}{325} \\
    \frac{\pxy{2}{1}}{\pxy{2}{0}} = \frac{350}{325} & \qquad \frac{\pxy{1}{1}}{\pxy{1}{0}} = \frac{630}{975}
\end{align*}

Supongamos que el costo de equivocarse es 1, entonces

\begin{equation*}
    \frac{L(0,1)}{L(1,0)} = \frac{1}{1} = 1
\end{equation*}

Con estos resultados obtener que el clasificador esta definido por:

si $x=1$

\begin{align*}
    \hat{y}(x) & = \mathbb{I} \left [\frac{\pxy{1}{1}}{\pxy{1}{0}} > \frac{L(0,1)P(Y=0)}{L(1,0)P(Y=1)}\right ] \\
               & = \mathbb{I} \left [\frac{420}{650}  > \frac{35}{65}\right ]                                  \\
               & = 1
\end{align*}

si $x=2$

\begin{align*}
    \hat{y}(x) & = \mathbb{I} \left [\frac{\pxy{2}{1}}{\pxy{2}{0}} > \frac{L(0,1)P(Y=0)}{L(1,0)P(Y=1)}\right ] \\
               & = \mathbb{I} \left [\frac{70}{65}  > \frac{35}{65}\right ]                                    \\
               & = 1
\end{align*}

si $x=3$

\begin{align*}
    \hat{y}(x) & = \mathbb{I} \left [\frac{\pxy{3}{1}}{\pxy{3}{0}} > \frac{L(0,1)P(Y=0)}{L(1,0)P(Y=1)}\right ] \\
               & = \mathbb{I} \left [\frac{175}{65}  > \frac{35}{65}\right ]                                   \\
               & = 1
\end{align*}

si $x=4$

\begin{align*}
    \hat{y}(x) & = \mathbb{I} \left [\frac{\pxy{4}{1}}{\pxy{4}{0}} > \frac{L(0,1)P(Y=0)}{L(1,0)P(Y=1)}\right ] \\
               & = \mathbb{I} \left [\frac{42}{65}  > \frac{35}{65}\right ]                                    \\
               & = 1
\end{align*}

Calculando el error se tiene lo siguiente:

Si $\hat{y}(x)=1$, entonces

\begin{equation*}
    E_{Y|X=x} [L(y,\hat{y}(x))] = L(0,1)P(Y=0) = P(Y=0) = 0.35
\end{equation*}

Supongamos que $L(y,\hat{y}(x))$ esta dada por

\begin{equation*}
    L(1,0) = 3 \qquad L(0,1) = 2
\end{equation*}

es decir, clasificar mal cuando el verdadero valor de y es 1 tiene un costo de 3, en otro caso tiene un costo de 2.

por lo tanto el clasificador esta dado por

si $x=1$

\begin{align*}
    \hat{y}(x) & = \mathbb{I} \left [\frac{\pxy{1}{1}}{\pxy{1}{0}} > \frac{L(0,1)P(Y=0)}{L(1,0)P(Y=1)}\right ] \\
               & = \mathbb{I} \left [\frac{420}{650}  > \frac{70}{195}\right ]                                 \\
               & = 1
\end{align*}

si $x=2$

\begin{align*}
    \hat{y}(x) & = \mathbb{I} \left [\frac{\pxy{2}{1}}{\pxy{2}{0}} > \frac{L(0,1)P(Y=0)}{L(1,0)P(Y=1)}\right ] \\
               & = \mathbb{I} \left [\frac{70}{65}  > \frac{70}{195}\right ]                                   \\
               & = 1
\end{align*}

si $x=3$

\begin{align*}
    \hat{y}(x) & = \mathbb{I} \left [\frac{\pxy{3}{1}}{\pxy{3}{0}} > \frac{L(0,1)P(Y=0)}{L(1,0)P(Y=1)}\right ] \\
               & = \mathbb{I} \left [\frac{175}{65}  > \frac{70}{195}\right ]                                  \\
               & = 1
\end{align*}

si $x=4$

\begin{align*}
    \hat{y}(x) & = \mathbb{I} \left [\frac{\pxy{4}{1}}{\pxy{4}{0}} > \frac{L(0,1)P(Y=0)}{L(1,0)P(Y=1)}\right ] \\
               & = \mathbb{I} \left [\frac{42}{65}  > \frac{70}{195}\right ]                                   \\
               & = 1
\end{align*}