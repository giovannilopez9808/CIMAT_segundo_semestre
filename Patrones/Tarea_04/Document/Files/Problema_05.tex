\section*{Problema 05}

\textbf{Es importante que las compañías de tarjetas de crédito puedan reconocer transacciones fraudulentas con tarjetas de crédito para que a los clientes no se les cobren artículos que no han comprado.}

\textbf{Descarga la base de datos `creditcard.csv' de la pagina Kaggle, \url{https://www.kaggle.com/mlg-ulb/creditcardfraud} Descripción de la base de datos: El conjunto de datos contiene transacciones realizadas con tarjetas de crédito en septiembre de 2013 por titulares de tarjetas europeos. Este conjunto de datos presenta transacciones que ocurrieron en dos días, donde tenemos 492 fraudes en 284,807 transacciones. El conjunto de datos está muy desequilibrado, la clase positiva (fraudes) representa 0, 172 \% de todas las transacciones. La base de datos solo contiene variables de entrada numéricas que son el resultado de una transformación PCA. Desafortunadamente, debido a problemas de confidencialidad, no se pueden proporcionar las características originales, ni más información general sobre los datos. Las características V\textsubscript{1}, V\textsubscript{2},..., V\textsubscript{28} son los principales componentes obtenidos con PCA, las únicas características que no han sido transformadas con PCA son `Tiempo' y `Cantidad' (`Time' and `Amount'). La función `Tiempo' (`Time') contiene los segundos transcurridos entre cada transacción y la primera transacción en el conjunto de datos. La función 'Cantidad' (`Amount') es la cantidad de la transacción. Caracteristica `Class' es la variable respuesta y toma valor 1 en caso de fraude y 0 en caso contrario.}

\textbf{Ejecuta 5 veces k-medias con diferentes semillas (random seeds) usando el esquema de inicialización `k-means++'.}

\subsection*{Punto 01}

\textbf{Muestra en una gráfica el mínimo de la función objetivo como una función de k, para $k \in \{2, 3, 4, 5, 6\}$}
\subsection*{Punto 02}

\textbf{Para cada procedimiento y para cada k, visualice la imagen comprimida usando la mejor corrida aleatoria. Cual es el menor valor de k para el cual está satisfecho con el resultado obtenido?}

Se selecciono el modelo que obtuviera la menor función objetivo dado el modelo y una k. En la figura se muestran las compresiones realizadas por cada modelo.

\begin{figure}[H]
    \centering
    \begin{subfigure}{17cm}
        \centering
        \includegraphics[width=17cm]{Graphics/Problema_04/cluster_2.png}
        \caption{K=2}
    \end{subfigure}
    \begin{subfigure}{17cm}
        \centering
        \includegraphics[width=17cm]{Graphics/Problema_04/cluster_4.png}
        \caption{K=4}
    \end{subfigure}
    \begin{subfigure}{17cm}
        \centering
        \includegraphics[width=17cm]{Graphics/Problema_04/cluster_8.png}
        \caption{K=8}
    \end{subfigure}
    \begin{subfigure}{17cm}
        \centering
        \includegraphics[width=17cm]{Graphics/Problema_04/cluster_16.png}
        \caption{K=16}
    \end{subfigure}
    \begin{subfigure}{17cm}
        \centering
        \includegraphics[width=17cm]{Graphics/Problema_04/cluster_32.png}
        \caption{K=32}
    \end{subfigure}
    \caption{Compresiones de imagenes para cada modelo y número de clusters dado.}
\end{figure}