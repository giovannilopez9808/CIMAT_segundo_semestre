\subsection*{Punto 01}

\textbf{Muestra en una gráfica el valor de la función objetivo de k-medias resultante sobre el conjunto de datos de entrenamiento como una función de k. Comenta lo que ves. Qué valor de k seleccionaría basándose solo en esta gráfica?}

En la figura \ref{fig:problema_03_train_scores} se muestran los resultados de la función objetivo objetivos con el número de clusters $k=2,3,\dots,15$. En esta gráfica se observa el comportamiento descendiente de la función objetivo conforme aumentan el número de clusters a tomar en cuenta. Este comportamiento aparenta llegar a una convergencia a un valor fijo conforme el número de clusters aumenta.

\begin{figure}[H]
    \centering
    \includegraphics[width=14cm]{Graphics/Problema_03/train_scores.png}
    \caption{Valor de la función objetivo para los datos de entrenamiento.}
    \label{fig:problema_03_train_scores}
\end{figure}

El valor de k que seleccionaria de la gráfica sería 3. Esto debido a que el score obtenido tiene una diferencia con su anterior k mayor a comparación de las demás diferencias obtenidas.