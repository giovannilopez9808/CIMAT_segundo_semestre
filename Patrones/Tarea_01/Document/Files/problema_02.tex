\section*{Problema 2}

\textbf{Supongamos que X=(X\textsubscript{1}, X\textsubscript{2}), Var(X\textsubscript{1})=Var(X\textsubscript{2})=1}.


\textbf{a) Supongamos que X\textsubscript{1} y X\textsubscript{2} son v.a. independientes con promedio 0. Verifica que cualquier dirección l da máxima varianza en las proyecciones.}

Como X\textsubscript{1} y X\textsubscript{2} son v.a independientes entonces, la matriz de covarianza Cov(X) es:

\begin{equation*}
    Cov(X) = \begin{pmatrix}
        1 & 0 \\
        0 & 1
    \end{pmatrix}
\end{equation*}

por lo que se obtiene que $Cov(X)= \mathbb{I}$. Entonces, se obtiene que:

\begin{align*}
    \underset{||l||}{max} \;\; \frac{l^t Cov(X) l}{l^t l} & = \underset{||l||}{max} \;\; \frac{l^t \mathbb{I} l}{l^t l} \\
                                                          & = \underset{||l||}{max} \;\; \frac{l^t l}{l^t l}            \\
                                                          & = \underset{||l||}{max} \;\; 1                              \\
                                                          & = \;\; 1
\end{align*}

por lo tanto, se maximiza la varianza para cualquier dirección de $l$ en las proyecciones.

\textbf{b) Supongamos que X\textsubscript{1} y X\textsubscript{2} son v.a. dependientes. Calcula la primer componente principal a mano. ¿Qué particularidad tiene?}

Suponiendo de la covarianza de X\textsubscript{1} Y X\textsubscript{2} es a, entonces, la matriz de covarianza es:

\begin{equation*}
    Cov(X) =
    \begin{pmatrix}
        1 & a \\
        a & 1
    \end{pmatrix}
\end{equation*}

Calculando la primer componente $l$, se obtiene que los valores propios de $Cov(X)$ es:

\begin{align*}
    |Cov(X)-\lambda \mathbb{I} | & = 0   \\
    \begin{vmatrix}
        1-\lambda & a          \\
        a         & 1- \lambda
    \end{vmatrix}    & =0    \\
    (1-\lambda)^2 -a^2           & =0    \\
    (1-\lambda-a )(1-\lambda+a)  & =0    \\
    \lambda_1                    & = 1-a \\
    \lambda_2                    & = 1+a
\end{align*}

Suponiendo que $a>0$, entonces $\lambda_2$ es el eigenvalor mayor. Calculando los vectores propios relacionados a $\lambda_2$, se obtiene que:

\begin{align*}
    \begin{pmatrix}
        -a & a  \\
        a  & -a
    \end{pmatrix} & = \begin{pmatrix}
        0 \\ 0
    \end{pmatrix}  \\
    \begin{pmatrix}
        -1 & 1  \\
        a  & -a
    \end{pmatrix} & = \begin{pmatrix}
        0 \\ 0
    \end{pmatrix}  \\
    \begin{pmatrix}
        -1 & 1 \\
        0  & 0
    \end{pmatrix} & = \begin{pmatrix}
        0 \\ 0
    \end{pmatrix} \\
    -x_1 + x_2                & = 0                          \\
    x_1                       & = x_2
\end{align*}

por lo tanto, el vector propio asociado a $\lambda_2$ es $v_2 = [x_1,x_1]^T$. La particularidad que tiene es que las componetes no tienen un valor determinado por lo que es necesario elegir el parámetro $x_1$ y en seguida normalizar el vector.