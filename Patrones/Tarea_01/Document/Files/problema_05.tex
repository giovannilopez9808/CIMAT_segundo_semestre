\section*{Problema 5}

\textbf{En este ejercicio usamos resultados del heptatlón feminina de los pasados juegos olímpicos de Tokyo (2021). En el archivo heptatlonTokyose pueden consultar los tiempos/distancias y el puntaje final (score) de 20 atletas.}

\textbf{a) Describe de manera general los datos sin considerar la columna con los puntajes finales, usando visualizaciones ilustrativas. Toma en cuenta que son pocas observaciones. Asi, no será posible llegar a conclusiones fuertes.}

\textbf{b) Un problema en un heptatlón es cómo convertir los resultados
    obtenidos en las diferentes pruebas en un puntaje final. Explo-
    ra la utilidad de PCA usando la proyección de los resultados de4
    las pruebas sobre el primer componente como una alternativa al
    puntaje final. ¿Cómo se relaciona con el puntaje final oficial?
    Información sobre cómo se calcula actualmente el puntaje: \href{http://theaftermatter.blogspot.mx/2012/06/maths-of-heptathlon-why-scoring-system.html}{Puntaje heptatlon}.}