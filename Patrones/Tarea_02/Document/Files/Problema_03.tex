\section*{Problema 2.3}

\textbf{Sea S un conjunto finito. Definimos como medida de similitud entre
    dos subconjuntos A y B de S:}

\begin{equation*}
    K(A,B) := \#(A \cap B)
\end{equation*}

\textbf{Busca una función tal que:}

\begin{equation*}
    K(A,B ) = \langle \Phi(A), \Phi(B)\rangle
\end{equation*}

Como $S$ es un conjunto finito, entonces podemos decir que el número total de elementos en $S$ es n. Dando asi que $S =\lbrace s_i \rbrace_{i=1}^n$. Sea $\Phi$ la siguiente función:

\begin{equation*}
    \Phi (X) = \sum_{i}^n \mathbb{I}_X(s_i)
\end{equation*}

donde $X$ es un conjunto (vector) de elementos y $\mathbb{I}_S$ una función indicadora tal que

\begin{equation*}
    \mathbb{I}_X(s_i) =
    \begin{cases}
        1 & \text{si }  s_i \in S    \\
        0 & \text{si } s_i \not\in S
    \end{cases}
\end{equation*}

Entonces,

\begin{align*}
    \langle \Phi(A) , \Phi(B)  \rangle & = \sum_{i=1}^n I_A(s_i) I_B(s_i) \\
                                       & = \sum_{i=1}^n I_{A\cap B}       \\
                                       & = \# A \cap B
\end{align*}

Lo anterior se pudo reducir ya que, la suma dará valores diferentes a cero solo si el elemento $s_i$ se encuentra en los dos conjuntos. Por lo tanto:

\begin{equation*}
    \Phi (X) = \sum_{i}^n \mathbb{I}_X(s_i) \qquad \mathbb{I}_X(s_i) =
    \begin{cases}
        1 & \text{si }  s_i \in X    \\
        0 & \text{si } s_i \not\in X
    \end{cases}
\end{equation*}