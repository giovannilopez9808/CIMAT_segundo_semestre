\section*{Problema 2.1}

\textbf{Sea $\{x i \}$ un conjunto de n vectores d dimensional. Definimos la matriz Kernel $[K_{i,j}]$ con $K_{i,j} = \left\langle x_i , x_j \right\rangle$ y $\mathbb{D}^2$ la matriz de distancias al cuadrada correspondiente. Verifica la identidad que usamos en clase:}

\begin{equation*}
    \mathbb{D}^2 = c1^t+1c^t-2\mathbb{X}\mathbb{X}^t
\end{equation*}

\textbf{con $1$ un vector de unos de longitud n y c el vector de longitud n con elementos $(\mathbb{K}_{i,i})^n_{i=1}$}

Sea X una matriz con elementos $x_{ij}$, entonces, la matriz $XX^T$ se puede escribir de la siguiente manera:

\begin{equation*}
    (XX^T)_{ij} = \sum_{k=1}^d x_{ik}x_{jk}
\end{equation*}

Con esto, el producto $1c^T$, se puede calcular como:

\begin{equation*}
    (1^Tc)_{ij} = \sum_{k=1}^d x_{jk}^2
\end{equation*}

De igual manera, el producto $c^T1$, se puede calcular como:

\begin{equation*}
    (c^T1)_{ij} = \sum_{k=1}^d x_{ik}^2
\end{equation*}

entonces el elemento $ij$ de la matriz $\mathbb{D}^2$, se obtiene lo siguiente:

\begin{align*}
    \mathbb{D}^2_{ij} & = \sum_{k=1}^d x_{ik}^2 - 2x_{ik} +x_{jk}^2 \\
                      & = \sum_{k=1}^d (x_{ik}-x_{jk})^2
\end{align*}

si tomamos $i=j$, se obtiene la diagonal de $\mathbb{D}^2$ es cero. Por lo tanto, la matriz $\mathbb{D}^2$ es la matriz de distancias entre los vectores $ij$.