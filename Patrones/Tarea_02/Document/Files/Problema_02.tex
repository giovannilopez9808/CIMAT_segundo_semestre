\section*{Problema 2.2}

\textbf{En la página 18 del archivo recpat6.pdf de la clase del 9 de febrero, verifica cómo que se obtiene la expresión $K_\Phi (x, y) = (1 + \langle x, y \rangle)^2$. De manera similar, supongamos que se define otro kernel K:}

\begin{equation*}
    K(x,y) = \langle x,y \rangle^3 x,y \in \mathbb{R}^2
\end{equation*}

\textbf{Busca una función $\Phi()$ tal que:}

\begin{equation*}
    K(x,y) = \langle \Phi(x), \Phi(y) \rangle
\end{equation*}

Sea $X=[x_1,x_2]^T$ y $Y=[y_1,y_2]^T$, calculando $\langle X,Y \rangle^3$ se obtiene lo siguiente:

\begin{align*}
    \langle X,Y \rangle^3 & = (x_1y_1+x_2y_2)^3                                                                                          \\
                          & = (x_1y_1)^3 + 3(x_1y_1)^2(x_2y_2) + 3(x_1y_1)(x_2y_2)^2 + (x_2y_2)^3                                        \\
                          & = x_1^3 y_1^3 + \sqrt{3}x_1^2 x_2 \sqrt{3}y_1y_2^2 + \sqrt{3}x_1x_2^2 \sqrt{3}y_1y_2^2+x_2^3y_2^3            \\
                          & = \langle (x_1^3,\sqrt{3}x_1^2x_2,\sqrt{3}x_1x_2^2,x^3),(y_1^3,\sqrt{3}y_1^2y_2,\sqrt{3}y_1y_2^2,y^3)\rangle \\
                          & =\langle \Phi(x) , \Phi(y) \rangle
\end{align*}

por lo tanto:

\begin{equation*}
    \Phi(z=(z_1,z_2)) = (z_1^3,\sqrt{3}z_1^2z_2,\sqrt{3}z_1z_2^2,z^3)
\end{equation*}